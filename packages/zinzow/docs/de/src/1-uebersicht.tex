\section{Systemübersicht}

Dieses Kapitel beschreibt den grundsätzlichen Aufbau des Systems, seine Hauptbestandteile sowie
deren Zusammenspiel. Ziel ist es, ein verständliches Gesamtbild zu vermitteln, ohne technische
Detailkenntnisse vorauszusetzen.

\subsection{Wichtige Links}

\begin{itemize}
    \item Directus (CMS): \url{https://cms.agrarservicenordost.de}
    \item Webseite: \url{https://agrarservicenordost.de}
    \item Quellcode: \url{https://github.com/CodesDoWork/monorepo/tree/master/packages/zinzow}.
\end{itemize}

\subsection{Grobe Funktionsweise des Systems}

Vereinfacht lässt sich das System wie folgt beschreiben:

\begin{itemize}
    \item Inhalte werden in einer Verwaltungsoberfläche (Directus) gepflegt.
    \item Diese Inhalte werden dauerhaft gespeichert.
    \item Die Webseite ruft die Inhalte ab und zeigt sie Besuchern an.
    \item Der Zugriff erfolgt sicher über das Internet.
\end{itemize}

\noindent Alle beteiligten Systeme laufen auf demselben Server, sind jedoch logisch voneinander
getrennt.

\subsection{Zentrale Komponenten}

\begin{description}
    \item[Datenbank (Postgres)]
    Die Datenbank speichert strukturierte Informationen wie Texte, Seiteninhalte, Konfigurationen
    und Benutzerinformationen. Sie enthält alle Inhalte, die über die Verwaltungsoberfläche gepflegt
    werden, mit Ausnahme hochgeladener Dateien wie Bilder oder Dokumente.

    \item[Content-Management-System (CMS) (Directus)]
    Directus stellt eine benutzerfreundliche Oberfläche zur Verfügung, über die Inhalte bearbeitet
    werden können. Es fungiert als Vermittler zwischen Datenbank und Nutzer/Webseite und stellt die
    Inhalte in einer standardisierten Form bereit.

    \item[Webseite]
    Die Webseite ist die sichtbare Oberfläche für Besucher. Sie fragt die benötigten Inhalte bei
    Directus an und bereitet diese für die Darstellung im Browser auf.

    \item[Cloudflare]
    Cloudflare ist ein externer Sicherheits- und Zugangsservice. Er stellt die öffentliche
    Verbindung über die Domain her und schützt das System vor unerwünschtem oder schädlichem
    Datenverkehr.

    \item[Cloudflare Tunnel]
    Der Tunnel verbindet Cloudflare sicher mit dem internen Server. Dadurch die Dienste auf dem
    Server nicht direkt aus dem Internet erreichbar sein.
\end{description}

\subsection{Zusammenspiel der Komponenten}

Ein typischer Zugriff verläuft in dieser Reihenfolge:

\begin{enumerate}
    \item Ein Besucher ruft die Webseite über die Domain auf.
    \item Cloudflare nimmt die Anfrage entgegen.
    \item Der Cloudflare Tunnel leitet die Anfrage intern an die Webseite weiter.
    \item Die Webseite fordert Inhalte von Directus an.
    \item Directus liest die Inhalte aus der Datenbank.
    \item Die Webseite zeigt die Inhalte im Browser an.
\end{enumerate}

\subsection{Grundlegende technische Konzepte}

Auch wenn sie im Hintergrund arbeiten, sind die folgenden Konzepte zentral für Verlässlichkeit und
Wartbarkeit des Systems.

\subsubsection{Container}

Jede Hauptkomponente läuft in einem eigenen abgeschlossenen Bereich (Container). Dadurch kann jede
Komponente unabhängig gestartet, gestoppt oder aktualisiert werden, ohne andere Teile zu
beeinträchtigen.

\subsubsection{Docker und Docker Compose}

Diese Werkzeuge sorgen dafür, dass alle Container gemeinsam und reproduzierbar betrieben werden
können. Sie definieren:
\begin{itemize}
    \item Welche Komponenten existieren.
    \item Wie sie miteinander verbunden sind.
    \item Welche Daten dauerhaft gespeichert werden.
\end{itemize}

\subsubsection{Volumes (dauerhafte Speicherung)}
\label{subsubsection:volumes}

Volumes sind Order, in denen wichtige Daten dauerhaft gespeichert werden. Sie sind \textbf{nicht} an
den laufenden Zustand einzelner Container gebunden und überdauern das Löschen oder Neustarten eines
Containers.

\noindent\textbf{Warum Volumes wichtig sind:}
\begin{itemize}
    \item Container können jederzeit neu gestartet oder ersetzt werden.
    \item Ohne Volumes würden Inhalte und Einstellungen dabei verloren gehen.
\end{itemize}

\noindent\textbf{Beispiele im System:}
\begin{description}
    \item[Datenbank-Volume (\texttt{postgres-data})]
    Speichert alle Inhalte der Datenbank dauerhaft. Auch bei Wartung oder Updates bleiben alle Daten
    erhalten.

    \item[Datei-Volume (\texttt{directus-uploads})]
    Enthält von hochgeladene Dateien wie Bilder oder Dokumente, die über Directus verwaltet werden.
\end{description}

\subsection{Zusammenfassung}

Das System trennt klar zwischen:
\begin{itemize}
    \item Inhaltspflege (Directus)
    \item Datenspeicherung (Datenbank und Volumes)
    \item Darstellung (Webseite)
    \item Sicherheit und Zugriff (Cloudflare)
\end{itemize}

Diese Struktur ermöglicht einen stabilen, sicheren Betrieb und erlaubt Änderungen oder Erweiterungen
einzelner Teile ohne Auswirkungen auf das Gesamtsystem.
