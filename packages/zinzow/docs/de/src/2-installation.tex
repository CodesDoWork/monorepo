\section{Installation}

Die „Installation“ der Software meint das Einrichten und Starten der Software auf einem Server. Dies
geschieht in folgenden Schritten:

\begin{enumerate}
    \item Systemanforderungen und sonstige Voraussetzungen prüfen.
    \item Konfigurationsdateien kopieren und anpassen.
    \item Docker Images kopieren.
    \item Container starten.
    \item Datenbank erstellen oder vorhandene Daten migrieren.
\end{enumerate}

\noindent Diese Schritte werden in den folgenden Abschnitten genauer beschrieben.

\subsection{Systemanforderungen und Voraussetzungen}

Bevor mit der Installation begonnen wird, müssen folgende Voraussetzungen erfüllt sein:

\begin{itemize}
    \item Es existiert ein Server mit mindestens:
    \begin{itemize}
        \item 500 MiB bis 1 GiB Arbeitsspeicher (RAM)
        \item ausreichendem Speicherplatz für:
        \begin{itemize}
            \item ca. 100 MB für die Datenbank
            \item zusätzlichen Speicherplatz für hochgeladene Bilder
        \end{itemize}
        \item einer stabilen Internetverbindung
        \item Docker und Docker Compose installiert.
    \end{itemize}
    \item Es gibt eine registrierte Domain (z.\,B. \texttt{argrarservicenordost.de}).
    \item Es gibt ein Cloudflare-Konto, das die Domain verwalten kann.
\end{itemize}

\subsubsection{Docker}

Die Software wird mit \textbf{Docker} betrieben. Docker sorgt dafür, dass alle Bestandteile der
Anwendung isoliert und zuverlässig laufen. Die Installationsanleitung gibt es hier:
\url{https://docs.docker.com/engine/install/}.

\subsubsection{Domain}

Domains kann man günstig bei \href{https://porkbun.com/}{Porkbun} mieten. Um sie bei Cloudflare
verüfgbar zu machen, müssen die richtigen Nameserver eingestellt werden. Das geht, indem man in der
Domain-Übersicht auf \textit{"NS"} klickt und dort Folgendes einträgt:

\begin{minted}[frame=lines]{text}
angela.ns.cloudflare.com. cameron.ns.cloudflare.com.
\end{minted}

Die zwei folgenden Bilder zeigen diesen Prozess:

\begin{figure}
    \centering
    \includegraphics[width=1\textwidth]{porkbun-overview.png}
    \caption{\textit{"NS"} klicken.}
\end{figure}

\begin{figure}
    \centering
    \includegraphics[width=0.5\textwidth]{porkbun-ns.png}
    \caption{Nameserver eintragen.}
\end{figure}

\subsubsection{Cloudflare}

Um diese Domain zu verwalten, muss es ein Cloudflare-Konto geben. Auf dem Cloudflare-Dashboard kann
dann die Domain hinzugefügt werden: \url{https://dash.cloudflare.com/}.

Zusätzlich muss ein sogenannter \textbf{Cloudflare Tunnel} eingerichtet sein. Dieser Tunnel
ermöglicht den sicheren Zugriff auf die Software, ohne dass die Dienste auf dem Server direkt aus
dem Internet erreichbar sind. Um einen Tunnel einzurichten, geht man wie folgt vor:

\begin{enumerate}
    \item Cloudflare Zero Trust öffnen und Account auswählen: \url{https://one.dash.cloudflare.com}.
    \item Auf \textit{Networks} → \textit{Connectors} klicken und dort \textit{Create a tunnel}
    auswählen.
    \item \textit{"Cloudflared"} wählen und einem Namen (beliebig) eingeben → weiter.
    \item Den Token aus dem Befehl kopieren (steht als letztes). Dieser wird später noch benötigt.
\end{enumerate}

\begin{figure}
    \centering
    \includegraphics[width=0.9\textwidth]{cloudflare-tunnel.png}
    \caption{Neuen Tunnel erstellen.}
\end{figure}

\subsection{Konfigurationsdateien vorbereiten}
\label{subsection:prepare-config-files}

Für den Betrieb werden zwei Dateien benötigt:

\begin{itemize}
    \item \texttt{docker-compose.yaml}
    \item \texttt{.env}
\end{itemize}

Die Datei \texttt{docker-compose.yaml} beschreibt, welche Container gestartet werden. Die Datei
\texttt{.env} enthält Konfigurationswerte wie Passwörter, Ports oder Tokens. Daher ist die
\texttt{.env} Datei ist sicher zu halten!

Die \texttt{docker-compose.yaml} befindet sich im Ordner der Quellcodes, während die \texttt{.env}
Datei eigenhändig erstellt werden muss. Letztere enthält pro Zeile eine Umgebungsvariable, um die
Software zu konfigurieren. Eine Zeile enhält immer \texttt{VARIABLE=WERT}, also z.\,B.
\texttt{DOMAIN=agrarservicenordost.de}.

\noindent Es müssen folgende Variablen gesetzt werden:

\begin{description}
    \item[STACK\_NAME] Ein Name für das Projekt.
    \item[WEBSITE\_DOMAIN] Die Domain (ohne \texttt{https://}).
    \item[WEBSITE\_URL] = \texttt{https://\$\{DOMAIN\}}
    \item[WEBSITE\_PORT] der Port, auf dem die Webseite im Container abrufbar ist (z.\,B. 4300).
    \item[WEBSITE\_SCALE] die Anzahl der Webseite-Container.
    \item[WEBSITE\_CMS\_TOKEN] der Static Token aus dem Directus User "Web Client", damit die
    Webseite auf
    \item[CMS\_SECRET] ein Secret, um Directus zu sichern.
    \item[CMS\_USERNAME] Eine Admin-Email für Directus.
    \item[CMS\_PASSWORD] Das Passwort des Directus Admin.  
    Directus zugreifen kann.
    \item[CMS\_DOMAIN] Domain des CMS (ohne \texttt{https://}). Wird über den Cloudflare Tunnel
    gesetzt (folgt später).
    \item[CMS\_URL] = \texttt{https://\$\{CMS\_DOMAIN\}}
    \item[POSTGRES\_USERNAME] der Benutzername der Datenbank.
    \item[POSTGRES\_PASSWORD] das Passwort des Datenbank-Benutzers.
    \item[SMTP\_HOST] der Host des SMTP Servers für den Mailversand.
    \item[SMTP\_PORT] der Port des SMTP Servers (meist 587).
    \item[SMTP\_USERNAME] der Benutzername für den SMTP Server.
    \item[SMTP\_PASSWORD] das Passwort für den SMTP Nutzer.
\end{description}

Beide Dateien werden in ein gemeinsames Verzeichnis auf dem Server kopiert.

\subsection{Docker Images kopieren}

Das Docker Image der Webseite wird nicht aus dem Internet geladen, sondern lokal bereitgestellt.

Vorgehensweise:
\begin{enumerate}
    \item Das Docker Image der Webseite wird lokal gebaut.
    \item Die Images werden als tar-Datei exportiert.
    \item Die tar-Datei wird auf den Server hochgeladen.
\end{enumerate}

Auf dem Server wird das Image entpackt und geladen:

\begin{verbatim}
docker load < image.tar
\end{verbatim}

Dieser Schritt stellt sicher, dass alle benötigten Komponenten lokal verfügbar sind.

\subsection{Container starten}
\label{subsection:run-containers}

Nachdem alle Dateien vorbereitet sind, können die Container gestartet werden. Dies geschieht mit
folgendem Befehl im Projektverzeichnis:

\begin{verbatim}
docker compose up -d
\end{verbatim}

\textbf{Erklärung:}
\begin{itemize}
    \item \texttt{docker compose} liest die Konfigurationsdateien.
    \item \texttt{up} startet alle Container.
    \item \texttt{-d} bedeutet, dass die Container im Hintergrund laufen.
\end{itemize}

\textbf{Hinweis}: Mit diesem Befehl können die Container nach ändern der Konfiguration neu gestartet
werden.

\subsection{Datenbank und Dateien vorbereiten}

Als nächstes müssen die Daten vorbereitet werden.

\subsubsection{Datenbank erstellen oder importieren}

Es gibt zwei Möglichkeiten:
\begin{itemize}
    \item Eine neue Datenbank in PostgreSQL anlegen
    \item Einen bestehenden Datenbank-Dump importieren
\end{itemize}

Neue Datenbank erstellen:
\begin{verbatim}
docker exec -it postgres psql -U $POSTGRES_USERNAME
CREATE DATABASE directus;
\end{verbatim}

Alternativ kann ein Dump importiert werden:

\begin{verbatim}
docker exec -i postgres psql -U $POSTGRES_USERNAME < dump.sql
\end{verbatim}

\subsubsection{Bilddateien kopieren}

Falls bereits Bilder existieren, müssen diese in den Directus-Container kopiert werden.

Beispiel:

\begin{verbatim}
docker cp ./bilder directus:/directus/uploads
\end{verbatim}

\textbf{Hinweis:} Der genaue Zielpfad kann je nach Konfiguration abweichen.

\subsection{Den Tunnel konfigurieren}

Sollte bis hierhin alles geklappt haben, müsste der in Cloudflare angelegte Tunnel nun als
\textit{"Healthy"} markiert werden. Zum Abschluss muss jetzt noch die Verbindung zu den Containern
hergestellt werden. Dazu muss Folgendes getan werden:

\begin{enumerate}
    \item In der \textit{Connectors} Übersicht auf den Tunnel klicken und dann auf \textit{"Edit"}.
    \item Zum Reiter \textit{"Published application routes"} navigieren und auf \textit{"Add a
    published application route"} klicken.
    \item Für die Webseite: Die Domain Auswählen und als Typ \texttt{HTTP} und als URL
    \texttt{website:4300} (oder welcher Port in der \textit{.env} steht) eingeben. Subdomain und
    Path leer lassen und speichern.
    \item Für das CMS wiederholen und als Subdomain \texttt{cms} eingeben und bei URL
    \texttt{cms:8055} eintragen. Damit ist die \texttt{CMS\_DOMAIN} in der \textit{.env}
    \texttt{cms.agrarservicenordost.de}.
\end{enumerate}

Die folgenden Bilder zeigen die einzelnen Schritte:

\begin{figure}[h]
    \centering
    \includegraphics[width=1\textwidth]{cloudflare-edit-tunnel.png}
    \caption{Tunnel bearbeiten.}
\end{figure}

\begin{figure}[h]
    \centering
    \includegraphics[width=0.9\textwidth]{cloudflare-tunnel-add-route.png}
    \caption{Neue Route anlegen.}
\end{figure}

\begin{figure}[h]
    \centering
    \includegraphics[width=0.9\textwidth]{cloudflare-setup-website.png}
    \caption{Webseite hinzufügen.}
\end{figure}

\begin{figure}[h]
    \centering
    \includegraphics[width=0.9\textwidth]{cloudflare-setup-cms.png}
    \caption{CMS hinzufügen.}
\end{figure}

Nach Abschluss dieses Schrittes ist die Software betriebsbereit.
