\section{Betrieb}

\subsection{Inhalte verwalten}

Um die Inhalte der Webseite zu pflegen, wird Directus genutzt. Die meisten Inhalte werden über den
\textit{Content}-Bereich (Würfel-Symbol) verwaltet. Einige Informationen werden allerdings auch in
den Einstellungen gepseichert. So ist in den Einstellungen (Zahnrad → EInstellungen) der Name und
die Beschreibung der Webseite ganz oben, sowie die Hintergrundfarben, der Copyright Claim, das
Gründungsjahr und der Unternehmensname ganz unten gespeichert. Im Erscheinungsbild (Zahnred →
Erscheinungsbild) kann die Primärfarbe eingestellt werden. Dateien befinden sich in der
Dateibibliothek (Ordner-Symbol).

Um die Änderungen auf der Webseite sichtbar zu machen, muss diese \textbf{zweimal} neu geladen
werden. Das liegt am Caching.

\subsubsection{Nutzer und Rollen}

Sollten Editoren hinzukommen, kann man das gute Rollensystem von directus nutzen:
\url{https://directus.io/docs/guides/auth/access-control}

\subsection{Andere Änderungen}

Um z.B. den Mail-Account oder den SMTP Server zu aktualisieren, müssen die entsprechenden Werte in
der \textit{.env} Datei auf dem Server geändert und die Webseite neu gestartet werden (siehe
\ref{subsection:prepare-config-files} und \ref{subsection:run-containers}).
